\documentclass[letterpaper,twocolumn,10pt]{article}
\usepackage{usenix2019_v3}
\begin{document}
\title{Computer Science 130, April 29th 2019 Debate Commentary}
\date{May 1st, 2019}
\author{{\rm Michael Wu}\\UCLA} % end author
\maketitle
\section{Introduction}

The following is my commentary on the debate focusing on whether sofware testing
and software development should be separate career paths.

\section{Pro Side, Presenter 1 (Alex)}

Alex began by defining various terms relevant to the debate. He explained what career paths were
and how jobs can be grouped into related clusters. He also gave a run down of the difference between
developers and testers. I found some of this explaining to be redundant and an ineffective use of time,
since it is pretty self apparent what a developer and what a tester is. I would have chosen to spend approximately
one third of the time defining terms and then the remainder on establishing the main arguments.

The first argument that Alex presented used an example of a developer named Singh who had to delay shipping
a product for months after development in order to learn how to test his software. He argued that a separate
testing department would streamline this process so that software would take less time to develop. This
argument could have been developed more with some statistics or examples. Additionally he could have mentioned
that having experienced testers ensures that the developer are held up to the same testing standards.
He was rushing since he spent too much time defining terms, and ended up running out of time. However,
his voice was clear throughout and he seemed fairly prepared.

\section{Con Side, Presenter 1 (Amit)}

Amit began by jumping quickly into his arguments which I liked. He argued that testers and developers should
be a combined role because developers know their code the best, so they can produce the best tests. He
also argued that when developers know that they must test their code, they write better since they know
which cases they need to account for. He also gave numerous industry examples such as Yahoo and Microsoft
who both found that combining developers and testers resulted in a better software development process.

This argument was very strong and he covered a lot of ground. He could have more explicitly addressed
the career path structure by saying that the move towards testing being done by developers makes it so
that a separate career path is impossible. This was something the pro side would later try to address, but
they did not do it too well. Amit also spoke quite fast and was reading off a script, so facing the audience
would have helped more. But overall he made good use of time and presented very well.

\section{Pro Side, Presenter 2 (Deven)}

Deven began with the additional argument that hiring testers as a separate role would save companies money
since testers can usually be paid less. He also spoke about splitting responsibility for the code amongst
the team. This point was confusing since it sounded like an argument for the con side, since teams would
be more confused about who should be trying to fix an error. What I think he was trying to say was that
since more people would work with a given test case, more people on the team would care about fixing the problem
and it would ensure a higher quality product.

Deven then began attacking the other sides presentation style and made some silly remarks. While somewhat entertaining,
I believe that he spent too much time on this to the detriment of his argument. He could have used his time
to cover more points, but by using this type of filler he diluted the effectiveness of his arguments. He attempted
to highlight how the other team was focusing on the jobs of a tester and developer, as opposed to the career
path structure. This did not make too much sense because if testers and developers were combined into one role,
there could not be separate career paths. He could have argued that while at a lower level, software engineers
need to know how to both test and develop, as they progress in their careers they could specialize into various
roles such as lead engineer, lead tester, and developer manager. This way companies could enjoy the benefits
of having developers test their own code while still ensuring there is a separate testing department that enforces
the quality of the tests. This would also reduce the amount of entry level testers so that the addition of a lead
tester role would have minimal overhead.

\section{Con Side, Presenter 2 (Michael)}

Michael responded to Deven's personal attacks with some short remarks and then addressed his last point by noting that
separate career paths imply that there will be separate testers and developers. This showed some quick thinking and made
a lot of sense. He then began talking about more industry examples to show how combining the tester and developer roles
would be beneficial. He stated that code coverage at Google increased by ten percent when developers tested their own code.
It also made the developers take more responsibility over their code. Since developers know how testers would think when
trying to test their code, they would catch more corner cases and write better code. He also responded to the pro side's
point that testers would be cheaper by stating that testers are underpaid. Companies could simply pay the testers more
and have them also do development for a minimal marginal cost.

I think that the con side's strategy of using lots of industry examples was an effective choice. Though they did not
introduce too many new arguments, they reinforced their existing points. The pro side had trouble answering such
overwhelming evidence.

\section{Pro Side, Presenter 3 (Ankur)}

Ankur reiterated some of the personal attacks that Deven made. He also repeated how separate career paths is a
different issue from combining individual jobs. He emphasized how having separate testers would be cheaper
and more cost effective. Overall he did not introduce many new arguments and was simply summarizing. He could
have tried to respond to the industry examples that the con side had given, but at this point his side seemed to lack
any new points to talk about. He could have said that even though developers know their code, they have a biased perspective
since they already have a mental model of the code in their head while a separate tester doesn't. So the tester may catch
problems that the developer has missed.

\section{Con Side, Presenter 3 (Amit)}

Amit wrapped up by noting that having developers and testers combined into one role reduces overhead. Additionally,
with the development of automated testing suites, developers can write tests and have them automatically run after building.
This reduces the need for a dedicated tester role. He finished off by mentioning Microsoft and Google again to show how
the industry has moved away from a separate testing role. This was very effective and a strong conclusion.

\section{Summary}

Overall, the con side seemed to present the better argument. Although at some parts the con side was reading off a script, they seemed well
prepared and could respond to anything the pro side brought up. They had many industry examples to back up their points and used
their time effectively. Although they spoke fast, they were able to think on their feet and cover a lot of ground. Both sides projected
their voices well and did not pause too much or stutter. The pro side could have used their time more effectively, reduced the personal
attacks, and lower the general amount of silliness. They could have emphasized how separate testers ensures an objective outside perspective
since developers may lack adequate testing skills if they are not fully focused on it.

\end{document}