\documentclass[12pt]{article}
\usepackage{chngcntr}
\usepackage{textcomp}
\usepackage{gensymb}
\usepackage{tabularx}
\usepackage{graphicx}
\usepackage{hyperref}
\counterwithin*{section}{part}
\title{CS 130, Homework 2}
\author{Will Fehrnstrom, Elizabeth Han, Michael Wang, Michael Wu\\
\\
UID: 504969404, 004815046, 804862406, 404751542}
\date{April 26th, 2019}
\begin{document}
\maketitle
\part{Cache Covert Channel Abuse Detection}

\section{Introduction}

\subsection{Purpose}

Online gaming is a rapidly growing industry with certain games having the potential to reach tens of
millions of monthly active users. With its ever-increasing appeal with a mainstream audience, it comes
as no surprise than an increasing number of users see merit in abusing the system to gain an undeserved
upper hand against others. In some cases these unfair advantages can be translated into monetary gain.
At the very least, cheating causes resentment and dissatisfaction with the game leading to reduced user
retention. As such, it is of great importance to gaming companies that cheaters be identified and prevented
from undermining the experience for others.

This document details the characteristics proposed for implementing a ReplayConfusion Architecture as
described by a study performed at the University of Illinois[2]. This system will be used to record
suspicious activity on the cloud gaming server which will then be replayed and analyzed to detect the
use of a cache covert channels. This will ideally allow system administrators to collect the evidence
necessary to identify this class of cheaters and take the appropriate actions ensuring the integrity of
the gaming application for all users.

\subsection{Scope}

This document is the System Requirements Specification for the proposed covert channel detection system.
The requirements cover the functional software components and nonfunctional general constraints the
developer will encounter when implementing the new functionality on the gaming server application.

\subsection{Intended Audience and Use}

This software requirements specification is intended for the client, developers, and end users who want
a preliminary overview of the proposed architecture for detecting cache covert channel abuse. It should
be noted that this document is the result of a cursory fifteen minute interview with the client and as
such is not a complete description of the system or the expected requirements. Developers should not
reference this document directly when implementing specific requirements. Rather, it serves as a very
high level overview of what the client initially envisioned the system to look like.

\subsection{Acronyms and Abbreviations}

As this document is intended for a general audience, acronyms and abbreviations will be kept to a minimum.

\section{Overall Description}

\subsection{Product Background and Perspectives}

There are a wide range of methods for cheaters to gain an advantage, each usually requiring its own specific
countermeasures to identify and address effectively. In this case it is believed that users are cheating via
cache covert channels. While users are supposedly limited to communicating with the server via channels that
we control so that only information the user should have access too is transmitted, covert channels allow an
alternate route for users to gain access to additional information the server does not intend them to have.
As the name implies, cache covert channels transmit this information through the cache. This usually occurs
when a trojan process infects the target server. A second malicious process on the outside known as the receiver
then primes the cache by filling it with data. The trojan selectively evicts the receiver’s data based on the
message the trojan wishes to send. Finally, the receiver reads the cache and computes the sender’s message
based on the number of cache misses.

Unfortunately “There is currently no general detector of cache-based covert channel attacks” [2]. However,
the University of Illinois study suggests a novel architecture with the potential to accurately detect a wide
range of cache covert channel attacks with “simple and non-intrusive” hardware requirements. Existing defences
against these attacks often involve isolating processes or introducing noise to the system muddling any
attempted transmissions. These methods alone are insufficient, failing to address all possible attacks or
adding significant overhead to the system. The study presents ReplayConfusion as a potential solution.
This method employs Record and deterministic Replay to detect covert channel attacks. It takes advantage
of the tendency for cache attacks to be highly reliant on “the specific mapping of addresses to cache
locations in the machine” and for cache conflicts resulting from these attacks to follow distinctive
patterns. By recording the non-deterministic events and cache resulting from a cache covert channel attack
and then deterministically replaying the recording using a “different mapping of addresses to cache locations”,
a new timeline of cache misses can be computed and compared with the original to identify any signs of an attack.

Methods for detecting suspicious behavior and punishing users found to have cheated are already in place.
This extension will simply serve as an additional tool for application administrators to collect evidence
and build a case against the accused, ensuring the fair treatment of all users.

\subsection{User Classes}

\begin{center}
    \begin{tabularx}{\linewidth}{|c|X|}
        \hline
        System Administrators & Administrators shall be able to record and replay activity performed on the
        server to determine if suspected users are guilty of committing cache covert channel attacks against
        the cloud gaming server.\\
        \hline
        Users & Normal users shall be able to use the application without knowing the existence of the
        ReplayConfusion architecture. Innocent users should never be falsely accused of cheating so false
        positives are unacceptable. Users will be able to report others as having suspicious behavior. Doing
        so will cause their match to be recorded and the accused to be flagged. Flagged individuals may then
        have their future games recorded.\\
        \hline
        Cheaters & Users attempting to perform cache covert channel attacks shall be detected and identified.
        As cheaters are likely to be repeat offenders it not paramount that every case of cheating be positively
        identified. However, the system should be accurate enough that users who are suspected of having performed
        a cache covert channel attacks on multiple occasions should be confirmed with very high regularity.\\
        \hline
    \end{tabularx}
\end{center}

\subsection{Assumptions and Dependencies}

The ReplayConfusion architecture shall interface seamlessly with the currently employed
HostelHosting\textsuperscript{\tiny\textregistered} servers Parts of the RnR (Record and deterministic Replay)
Module and Performance Monitoring Unit are already built within the existing system and shall be extended rather
than replaced in the new architecture.

\subsection{Constraints}

\begin{enumerate}
    \item \textbf{Hardware} The system shall be able to run concurrently with gaming applications on cheap
    multitenanted cloud servers. As such, it shall introduce little overhead to the server and shall not affect
    the latency of the gaming applications.
    \item \textbf{Availability} It is not paramount that every cheater be caught.The system should be able to
    operate with few interruptions but this constraint is secondary to minimizing performance overheads.
    \item \textbf{Security} As recordings of games are freely viewable by normal users, it is not the greatest
    concern that the system secure the secrecy of the data it stores.
\end{enumerate}

\subsection{High Level System Decomposition}

The diagram below representing the system was copied directly from the the University of Illinois study. In
our case the Recorded Apps will be the company’s gaming applications. “The shaded boxes show the components
of the ReplayConfusion Architecture” [2]. The Cache Address Computation Unit matches the cache with physical
addresses. The Cache Configuration Manager allows a system administrator to choose an alternative mapping for
the cache address computation unit such that during normal operations the original mapping is used while during
replay an alternative mapping unique to the process is used. The Cache Profile Manager collects data of the
cache such as number of instructions, accesses, and misses, producing a cache timeline which may be stored
and analyzed. The RnR (Record and deterministic Replay) module records non-deterministic inputs to be replayed
in a new cache address mapping.

\begin{figure}
  \includegraphics[width=\linewidth]{architecture.png}
  \caption{High-level ReplayConfusion architecture, copied from source 2}
  \label{fig:boat1}
\end{figure}

\section {System Features \& Requirements}

\subsection{Functional Requirements}

\subsubsection{General}

\begin{enumerate}
    \item Functional Requirement 1 (ID: GFR1)
    \begin{enumerate}
        \item Title: Recording Module
        \item Required For: Basic Use and Functionality
        \item Description: The existing recording module shall be adapted such that all non-deterministic
        inputs to the system necessary to perform ReplayConfusion are recorded and stored at a later time.
        The module shall record the activity of all users during a game which will be discarded after a short
        time following the game’s conclusion unless a user is reported for suspicious behavior, either by
        another user or by already existing cheating-detection software. The recordings of individuals flagged
        as suspicious shall be held for a longer period of time, potentially indefinitely.
        \item Dependencies: The HostelHost\textsuperscript{\tiny\textregistered} server recording module is
        functioning normally
    \end{enumerate}
    \item Functional Requirement 2 (ID: GFR2)
    \begin{enumerate}
        \item Title: Cache Configuration Module
        \item Required For: Basic Use and Functionality
        \item Description: A Cache Configuration Manager OS module shall be produced which employs the
        HostelHost\textsuperscript{\tiny\textregistered} server Cache Address Computation Unit allowing application
        administrators to provide a process-unique mapping between cache and hardware addresses on demand. The mapping
        shall have a sufficiently large domain space such that it is highly unlikely that a given cache covert channel
        attack recording replayed on two randomly produced cache mappings will result in similar cache profiles
        \item Dependencies: The HostelHost\textsuperscript{\tiny\textregistered} server Cache Address Computation is
        functioning normally
    \end{enumerate}
    \item Functional Requirement 3 (ID: GFR3)
    \begin{enumerate}
        \item Title: Profile Manager
        \item Required For: Basic Use and Functionality
        \item Description: A Cache Profile Manager OS module shall be produced which employs the
        HostelHost\textsuperscript{\tiny\textregistered} server Performance Monitoring Unit to record all relevant
        activity necessary to perform ReplayConfusion and store it for later
        \item Dependencies: The HostelHost\textsuperscript{\tiny\textregistered} server Performance Monitoring Uni
        is functioning normally
    \end{enumerate}
    \item Functional Requirement 4 (ID: GFR4)
    \begin{enumerate}
        \item Title: Cache Analyzer
        \item Required For: Basic Use and Functionality
        \item Description:  A program shall be produced which takes as input two cache timelines as produced
        by ReplayConfusion and outputs how likely it is that they are the result of a cache covert channel attack.
        \item Dependencies: ReplayConfusion has been implemented correctly.
    \end{enumerate}
\end{enumerate}

\subsubsection{System Administrators}

\begin{enumerate}
    \item Functional Requirement 1 (ID: AFR1)
    \begin{enumerate}
        \item Title: Administrator Access
        \item Required For: Basic Use and Functionality
        \item Description: System Administrators shall have full access to the features described in this document
        \item Dependencies: None
    \end{enumerate}
\end{enumerate}

\subsection{External Interface Requirements}

\subsubsection{User Interface}

The interface of the normal user shall be unaffected by the addition of this functionality. Interfaces for system
administrators have no requirements so long as all features described in this document are viewable.

\subsubsection{Software Interface}

The proposed architecture shall interface seamlessly with the already in place gaming software.

\subsubsection{Hardware Interface}

The proposed architecture shall interface with the already in place server architecture, with the recording module
being extended and Cache Profile Manager operating on top of the existing Performance Monitoring Unit.

\subsubsection{Communications Interface}

No specification was made regarding the development language, essential libraries, database, or programming
environment to be used with the intent being to allow the developers to determine the most cost-effective implementation.

\subsection{System Features}

\subsubsection{Cache Time-line Generation}

\begin{enumerate}
    \item Description \& Priority\\
    Priority 1: System Administrators shall be able to record the non-deterministic inputs to the system resulting
    from a specific user and produce simultaneously produce a time-line of the corresponding cache activity
    \item Stimulus \& Response Sequences
    \begin{itemize}
        \item User has been flagged as suspicious for any reason.
        \item Record and Deterministic Replay module begins recording non-deterministic inputs to the gaming
        application originating from the flagged user
        \item The Cache profile manager produces a history of the cache
        \item A cache timeline is produced.
    \end{itemize}
\end{enumerate}

\subsubsection{Cache Configured Replay}

\begin{enumerate}
    \item Description \& Priority\\
    Priority 2: System Administrators shall be able to replay recordings on a reconfigured cache to produce
    an alternate cache time-line
    \item Stimulus \& Response Sequences
    \begin{itemize}
        \item The system administrator identifies the process recorded and the cache configuration manager
        automatically computes an alternate cache address mapping.
        \item The recording is replayed on the new address mapping
        \item An alternate cache timeline is produced
    \end{itemize}
\end{enumerate}

\subsubsection{Covert Channel Attack Determination}

\begin{enumerate}
    \item Description \& Priority\\
    Priority 2: System Administrators shall be able to compare two timelines produced from the same non-deterministic
    inputs under different cache configuration to receive the probability of the user having performed a cache
    covert channel attack.
    \item Stimulus \& Response Sequences
    \begin{itemize}
        \item The system produces two such timelines and feeds it to the timeline comparison program
        \item The program outputs the probability
    \end{itemize}
\end{enumerate}

\subsection{Non-Functional Requirements}

\begin{enumerate}
    \item Non-functional Requirement 1 (ID: NR1) \begin{enumerate}
        \item Title: Latency Requirements
        \item Required For: User Experience
        \item Description: The Record and deterministic Replay module system shall not introduce a noticeable
        increase in latency for users of the gaming application. Other modules are expected to be used during
        times of low activity and can be allowed to consume more resources.
        \item Dependencies: A minimum viable RnR module has been implemented and is ready for testing.
    \end{enumerate}
    \item Non-functional Requirement  2(ID: NR2) \begin{enumerate}
        \item Title: Error Rate Requirement
        \item Required For: Basic Use and Functionality
        \item Description:  False positives rate should be limited to 0.001\% of reported positives. False
        negatives shall not exceed 80\% of actual positive readings
        \item Dependencies: A minimum viable ReplayConfusion system has been implemented and is ready for testing
    \end{enumerate}
\end{enumerate}

\section{Change Management}

The client and development team shall meet once every two weeks during which the client will have the opportunity
to express any desire to make changes to the requirements. After a review of the potential effects on the system
and the likely cost of implementing the changes, the client will have the opportunity to retract the proposal or
proceed at which point the appropriate adjustments to the requirements specification will be implemented. The client
has expressed a (potentially unreasonably) high degree of confidence in these requirements and has suggested that
any changes presented in the future will only come as the result of an extreme emergency. Such changes should
therefore be regarded as paramount and take precedence over the requirements expressed in this document.

\section{Document Approvals}

\begin{center}
    \begin{tabular}{|c|c|c|}
        \hline
        Signatory Name & Signatory Position & Date \\ \hline
        Michael Wang & Client & 4/25/19 \\ \hline
        Elizabeth Han & Software Developer & 4/25/19 \\ \hline
        William Fehrnstrom & Software Developer & 4/25/19 \\ \hline
        Michael Wu & Software Developer & 4/25/19 \\ \hline
    \end{tabular}
\end{center}

\section{Supporting Information}

\begin{enumerate}
    \item Description of Cache Covert Channels from: \url{https://cmaurice.fr/pdf/ndss17_maurice.pdf}
    \item Description of ReplayConfusion System from: \url{https://iacoma.cs.uiuc.edu/iacoma-papers/micro16_1.pdf}
\end{enumerate}

\pagebreak
\part{Social Media Parental Controls}

\section{Introduction}

\subsection{Purpose}

This system is an extension to the functionality of the social media site FaceTime. It implements various
optional parental controls providing parents increased awareness and control of their children’s account activity.

This document is merely a brief overview of the functionality requested by the client and was compiled over the
course of a brief fifteen-minute interview. Developers may refer to this document for a high-level summary of the
system but should not rely on it when implementing specific requirements. Concerned parents will find an
approachable description of the functions initially envisioned for the system.

\subsection{Scope}

This document System Requirements Specification for the requested parental controls system. The requirements
cover the functional software components and nonfunctional general constraints the developer will encounter
when implementing the new functionality on the FaceTime social media platform.

The parent must first have a FaceTime account to link to the child’s account. From there the parent can
choose among a variety of safety features such as content filtering, account curfews, and automatic alerts
in response to friend requests, invitations to chat, etc. Management of the child’s safety features is performed
via the parent’s account.

\subsection{Intended Audience and Use}

Concerned Parents and their children shall benefit from the increased oversight afforded to parents of
their children’s FaceTime accounts. System Administrators shall monitor the new system and implement any
changes and improvements when necessary.

\subsection{Acronyms and Abbreviations}

\begin{enumerate}
    \item FaceTime is the world’s leading social media platform. It is the owner and developer of the system
    described in this document.
\end{enumerate}

\section{Overall Description}

\subsection{Product Background and Perspectives}

With the saturation of social media within everyday life, it has become a near necessity for most individuals to
have an account to interact with those around them. Whether it’s to chat with friends or catch up on the news, users
often find themselves spending a significant portion of their day on these sites. However, many parents are concerned
that younger users may be at risk when interacting with others over the service. They fear that online predators and
the like may attempt to take advantage of children, tricking them into revealing private information or luring them
into unsafe situations. In response, FaceTime presents a comprehensive suite of parental controls placing the power
and responsibility of safe use of FaceTime by children entirely in the hands of parents.

\subsection{User Classes}

\begin{center}
    \begin{tabularx}{\linewidth}{|c|X|}
        \hline
        System Administrators & These users will collect metadata on the new system to help improve the user
        experience. \\
        \hline
        Young Users & Users  with FaceTime accounts under 18 who are at risk of unsafe online social interactions.
        They will be placed under the protection of a parent user.\\
        \hline
        Parents & Users with children with FaceTime accounts. They will be given access to parental controls which
        help monitor and protect the child’s activity on FaceTime.\\
        \hline
    \end{tabularx}
\end{center}

\subsection{Assumptions and Dependencies}

It is assumed that both parent and child have FaceTime accounts. This system is built upon an already existing
and very well established FaceTime framework. Any requirements involving changes to the underlying framework are
naturally significantly more expensive than requirements that can be built on top of it. As a system built within
FaceTime, only activity on a FaceTime account can be monitored. We have no jurisdiction over your child’s activity
on other sites such a Snapchat or MySpace. Strangers your child encounters in real life are likewise unaffected.

\subsection{Constraints}

\begin{enumerate}
    \item The system must be available on all devices and Operating systems which can run FaceTime
    \item Tee system must not adversely affect the availability or reliability of the FaceTime platform overheads.
    \item The system must not be abusable such that an any user can exercise inappropriately gained parental
    controls on another arbitrary user.
    \item The new features must be available to all users simultaneously upon release.
\end{enumerate}

\section {System Features \& Requirements}

\subsection{Functional Requirements}

\subsubsection{System Administrators}

\begin{enumerate}
    \item Functional Requirement 1 (ID: AFR1)
    \begin{enumerate}
        \item Title: Administrator Access
        \item Required For: User Feedback
        \item Description: System administrators shall have access to aggregate data about about regarding use of
        the system. This data shall be aggregated from a large enough dataset as to not compromise the privacy of
        any individual. For example, administrators may wish to query the average curfew times put in place for
        children under 16.
        \item Dependencies: None
    \end{enumerate}
    \item Functional Requirement 2 (ID: AFR2)
    \begin{enumerate}
        \item Title: Administrative Termination
        \item Required For: Third Party Intervention
        \item System administrators shall be reserved the right to terminate support for any parental controls for
        any reason at any time. This includes removing parental controls for specific individuals or for entire
        principalities.
        \item Dependencies: None
    \end{enumerate}
\end{enumerate}

\subsubsection{Parents}

\begin{enumerate}
    \item Functional Requirement 3 (ID: PFR1)
    \begin{enumerate}
        \item Title: Parent Request Initiation
        \item Required For: Basic Use and Functionality
        \item Description: Users over 18 referred to as the parent shall have the option to send a “Parent request”
        to another user under 18 referred to as the child. The child can accept this request much like a friend
        request. Doing so grants the parent access to the parental controls described in this document.
        \item Dependencies: Both parent and child have and are operating on their own FaceTime accounts
    \end{enumerate}
    \item Functional Requirement 4 (ID: PFR2)
    \begin{enumerate}
        \item Title: Parent Termination
        \item Required For: Basic Use and Functionality
        \item Description: The parent shall be able to terminate the parent relationship at any time.
        \item Dependencies: The parent and child have a current parent-child relationship
    \end{enumerate}
    \item Functional Requirement 5 (ID: PFR3)
    \begin{enumerate}
        \item Title: Curfew Controls
        \item Required For: Control Enforcement
        \item Description: The parent shall be able to terminate the parent relationship at any time.
        \item Dependencies: The parent and child have a current parent-child relationship
    \end{enumerate}
    \item Functional Requirement 6 (ID: PFR4)
    \begin{enumerate}
        \item Title: Parent newsfeed access
        \item Required For: Monitoring
        \item Description: The parent shall have access to the child’s newsfeed and the home pages of all users the
        child has access to.
        \item Dependencies: The parent and child have a current parent-child relationship
    \end{enumerate}
    \item Functional Requirement 7 (ID: PFR5)
    \begin{enumerate}
        \item Title: Parent message access
        \item Required For: Monitoring
        \item Description: The parent shall have access to the child’s newsfeed and the home pages of all users
        the child has access to.
        \item Dependencies: The parent and child have a current parent-child relationship
    \end{enumerate}
    \item Functional Requirement 8 (ID: PFR6)
    \begin{enumerate}
        \item Title: Parent message access
        \item Required For: Monitoring
        \item Description: The parent shall be able to view all friend requests and invitations to message being
        sent to and from the child. Additionally, the parent will be alerted each time the child receives a message.
        The contents of this message shall not be viewable.
        \item Dependencies: The parent and child have a current parent-child relationship
    \end{enumerate}
    \item Functional Requirement 9 (ID: PFR7)
    \begin{enumerate}
        \item Title: Content Filtering
        \item Required For: Control Enforcement
        \item Description: The parent shall be able to place a filter on the child’s account which will automatically
        remove certain words or phrases.
        \item Dependencies: The parent and child have a current parent-child relationship
    \end{enumerate}
\end{enumerate}

\subsubsection{Children}

\begin{enumerate}
    \item Functional Requirement 10 (ID: CFR1)
    \begin{enumerate}
        \item Title: Accepting Parent Request
        \item Required For: Basic Use and Functionality
        \item Description: he child can accept this request much like a friend request. Doing so grants the parent
        access to the parental controls described in this document. As the parental relationship is vastly different
        from the friend relationship, multiple warning screens will be presented to anyone attempting to accept a
        parent request.
        \item Dependencies: Both parent and child have and are operating on their own FaceTime accounts
    \end{enumerate}
    \item Functional Requirement 11 (ID: CFR2)
    \begin{enumerate}
        \item Title: Child Terminating Relationship
        \item Required For: Basic Use and Functionality
        \item Description:Once the child turns 18 they shall from that point onward have the option to terminate
        the parent relationship.
        \item Dependencies: None
    \end{enumerate}
\end{enumerate}

\subsection{External Interface Requirements}

\subsubsection{User Interface}

The child’s interface shall be unaffected by the presence of a parent relationship accept when parental controls
such as curfews are being enforced. The parent’s interface shall include intuitive controls that are inline with
FaceTime’s design philosophy and should reuse user interface designs when possible. For example, notifications
indicating the child has received a friend request shall appear similar to a normal friend request.

\subsubsection{Software Interface}

The software must be integrated seamlessly into the FaceTime  platform. No previous functionality shall be impeded
by the addition of these features.

\subsubsection{Hardware Interface}

The additional functionality must be enclosed in and interface with the same servers and databases already being
used by FaceTime.

\subsubsection{Communications Interface}

Specifications regarding the development language, libraries, database, or programming environment are left
flexible to encourage developers to find the best possible implementation. However, developers should default to using
the tools employed in the development of similar features in the rest of the FaceTime feature suite.

\subsection{System Features}

\subsubsection{Child Connection and Control}

\begin{enumerate}
    \item Description \& Priority\\
    Priority 1: The Parent shall be able to set up a parent-child relationship with a child and initiate parental controls.
    \item Stimulus \& Response Sequences
    \begin{itemize}
        \item The parent logs into their FaceTime account and sends a parent request to a child user.
        \item The child logs into their FaceTime account and accepts the parent request. The child user will be
        presented with multiple warnings explaining clearly the risks involved in accepting the request.
        \item The parent on their FaceTime account gains access to a new set of settings allowing for monitoring
        and parental control of the child's account.
        \item The parent receives notifications allowing for passive monitoring of any attempts to communicate with
        the child over FaceTime.
    \end{itemize}
\end{enumerate}

\subsection{Non-Functional Requirements}

\begin{enumerate}
    \item Non-functional Requirement 1 (ID: NR1)
    \begin{enumerate}
        \item Title: Reliability Requirements
        \item Required For: User Experience
        \item Description: The reliability of the FaceTime platform shall be unaffected by the addition of these
        features. As this system is merely a simple extension of already existing features, it is expected that
        failure in the parental controls would be equivalent to failure of the entire FaceTime system.
        \item Dependencies: A minimum system has been implemented and is ready for testing.
    \end{enumerate}
    \item Non-functional Requirement 2 (ID: NR2)
    \begin{enumerate}
        \item Title: Security Requirement
        \item Required For: Security
        \item Description:  The security of user’s data shall be unaffected by the addition of these features
        with the acknowledgement that the parent may unintentionally leak the private information of their child.
        \item Dependencies: A minimum viable system has been implemented and is ready for testing
    \end{enumerate}
    \item Non-functional Requirement 3 (ID: NR3)
    \begin{enumerate}
        \item Title: Privacy Requirement
        \item Required For: Privacy
        \item Description:  The system shall not reveal any information about the child’s account that has not been
        explicitly stated by this document. In particular, the parent shall be have login permissions to the child’s account.
        \item Dependencies: A minimum viable system has been implemented and is ready for testing
    \end{enumerate}
    \item Non-functional Requirement 4 (ID: NR4)
    \begin{enumerate}
        \item Title: Legality  Requirement
        \item Required For: Legal Compliance
        \item Description:  The legality of certain features may be contested in some principalities. Certain controls
        shall be able to be disabled in such principalities without affecting the functionality of other controls
        \item Dependencies: A minimum viable system has been implemented and is ready for testing
    \end{enumerate}
    \item Non-functional Requirement 5 (ID: NR5)
    \begin{enumerate}
        \item Title: Control Enforcement Latency
        \item Required For: User Experience
        \item Description:  controls such as curfews and content filters shall be enforced before any content is made
        viewable on the child’s device.
        \item Dependencies: A minimum viable system has been implemented and is ready for testing
    \end{enumerate}
    \item Non-functional Requirement 6 (ID: NR6)
    \begin{enumerate}
        \item Title: Notifications Latency
        \item Required For: User Experience
        \item Description:   notifications to the parent shall occur no later than 1 second after the child receives
        the notification.
        \item Dependencies: A minimum viable system has been implemented and is ready for testing
    \end{enumerate}
\end{enumerate}

\section{Change Management}

The client shall receive semi-weekly updates of the progress of the system, during which the client will have the
opportunity to express any desire to make changes to the requirements. After a review of the potential effects on
the system and the likely cost of implementing the changes, the client will have the opportunity to retract the
proposal or proceed at which point the appropriate adjustments to the requirements specification will be implemented.
Cost is not a major concern to the client so such reviews shall focus mainly on guaranteeing the feasibility of the
changes rather than computing the minutiae of resource requirements.

\section{Document Approvals}

\begin{center}
    \begin{tabular}{|c|c|c|}
        \hline
        Signatory Name & Signatory Position & Date \\ \hline
        Michael Wang & Client & 4/25/19 \\ \hline
        Elizabeth Han & Software Developer & 4/25/19 \\ \hline
        William Fehrnstrom & Software Developer & 4/25/19 \\ \hline
        Michael Wu & Software Developer & 4/25/19 \\ \hline
    \end{tabular}
\end{center}

\pagebreak
\part{Secure Classified File Sharing}

\section{Introduction}

\subsection{Purpose}

The purpose of this file sharing service is the following: service FBI agent and officials requests'
in a highly authenticated manner; using many-factor authentication in order to share confidential case
files between FBI offices. There is a distinct need to share files in a more efficient manner than by
shipping paper copies, which are vulnerable to theft, loss, and tampering. The FBI's internal communication
channels are already in place and are secure: therefore, we need to plug into those communication channels
in order to send our files. Servers will be distributed across different field offices to prevent any single
point of attack. Furthermore, the files will primarily still be kept on paper, but will be shared by scanning,
and then deleted after sending.

\subsection{Scope}

This project will focus on the sharing of currently on-paper reports, with an emphasis on the access control
required to do so. Specifically out of scope are any other messages on the platform other than the distribution
of official documents. Further, we are targeting field offices, so this project is not expected to conform to
any other office requirements. The physical infrastructure that we will put our software on is explicitly in
scope at the clients’ request. In fact, we are responsible for procuring that hardware. In addition, the methods
of agent authentication in order to scan documents and transfer them are in scope-we will find a suitable, PC
compatible iris scanner.

\subsection{Intended Audience and Use}

The FBI office staff, officers, and agents. This software specification should also be viewable by any penetration
testers. That is, allowing them to view this is beneficial in exposing system vulnerabilities.

\subsection{Acronyms and Abbreviations}

\begin{enumerate}
    \item FBI: Federal Bureau of Investigation, Government Agency for prosecuting US internal affairs.
    \item Agent: a highly authorized human entity tasked with carrying out the FBI's mission.
    \item IT: Internet Technology
    \item AES: Advanced Encryption Suite
    \item Client: The FBI agency.
    \item Development Team: The contracted development team, working on this product.
\end{enumerate}

\section{Overall Description}

\subsection{Product Background and Perspectives}

Previously, FBI offices in the field were prevented from effectively coordinating with each other on one case
because of the difficulties in distributing the case information (on paper) necessary. The intent of this software
product is to solve this problem by digitizing papers for a short period of time for distribution. As the problem
domain involves foreign intelligence agencies and officers charged with penetrating the US’s internal monitoring
system, the FBI, we should expect that any flaw in the software that we place will ultimately be exploited by the
adversary. Therefore, security is of the highest importance. We must ensure that only officers assigned to a
specific case are allowed to send and receive documents relating to that case. Further, documents should be
distributed to all offices involved in a case at their request, and not be distributed to offices that have no
part in a case. We must plan for negligence, i.e. officers leaving our software open in order to deal with some
more pressing matter. No matter the organization, office workers will tend to let their guard down.

\subsection{User Needs}

\begin{center}
    \begin{tabularx}{\linewidth}{|c|X|}
        \hline
        FBI Agent & This user class comprises all of the FBI agents that will be using our product in the field to
        help distribute case information between offices.\\
        \hline
        Government officials & Periodically, government officials outside the FBI department may wish to see this
        software system in use as part of an audit. Appropriate security procedures should be followed in this
        situation to provide minimal access.\\
        \hline
        Foreign Operatives & This user class is a user class in a very non-traditional sense: they will likely
        examine our product closely to find any exploits: planning is required to make exploiting the system as
        difficult as possible for this user class.\\
        \hline
        FBI IT Management & The FBI's internal IT management team will need to maintain the system after we've built
        it, thus they will invariably become the primary consumers of our documentation created during the development
        process, and they will have to have the most intricate understanding of the hardware, software, and communications
        interconnection. We will be providing to them all our research and development documents, and further, we would
        like to allow auditing of the system response to user load, without giving away any confidential information.\\
        \hline
    \end{tabularx}
\end{center}

\subsection{Assumptions and Dependencies}

An assumption that cannot be made about our operating environment is that it will be secure. We must assume that
any bad actor can at any time gain physical access to our system. We assume that any agent authenticated by our
authentication scheme is a good actor: it is the job of the authentication equipment to ensure that this is the case.
Furthermore, we assume that the installation environment of the system will be a good one-that is, no bad actors will
be present for the installation process. This is a key assumption, but one that must be taken if we are to roll out
the system at all. We assume that the client’s requests are in good faith, and that they too are good actors. Furthermore,
since our own developers are subject to intense scrutiny, we assume that they too are good. We depend heavily on the
hardware platform that we use. We also depend on the electrical systems currently place in FBI field offices, and the
internal wired network that the FBI uses in order to communicate and fetch information for the wider web. We also
depend on the transportation infrastructure used to put the hardware in place.

\subsection{Constraints}

\begin{enumerate}
    \item A minimum number of external intrusions must be allowed.
    \item Document sharing must be achievable by one entity with a network connection, provided they have appropriate
    authorization.
    \item The system must be portable to a wide variety of locations with varying network access.
    \item The system must be recoverable in the event of electrical failure.
    \item Multi-factor authentication must be used.
    \item The system must only be used within FBI field offices.
    \item Authentication classes should be hardwired, and unconfigurable.
    \item The hardware required per office must take up no more than 10 cubic feet.
    \item The hardware required per office must be able to be shipped in a variety of different formats (it must be durable).
\end{enumerate}

\section{System Features \& Requirements}

\subsection{Functional Requirements}

Unless otherwise stated, every functional requirement have the following dependencies:
\begin{enumerate}
    \item Transport Network
    \item Electrical Network
    \item FBI Communications infrastructure
\end{enumerate}
\textbf{Priorities are ranked from level 1 (PR1) (imperative) to level 3 (PR3) (Add-On Feature)}

\subsubsection{User Class A - FBI Agent}

\begin{enumerate}
    \item Functional Requirement 1 (ID: AFR1) (PR2)
    \begin{enumerate}
        \item Title: Accessibility
        \item Required For: Basic Use and Functionality
        \item Description: The software system MUST be accessible directly from any field agent’s desk.
        The authentication systems used to permit use of the system must be similarly accessible, or be
        multiplied to an extent that allows direct use. The system must not be accessible to unintended users.
    \end{enumerate}
    \item Functional Requirement 2 (ID: AFR2) (PR1)
    \begin{enumerate}
        \item Title: Authentication
        \item Required For: Basic Use and Functionality
        \item Description: The software system MUST not be usable without first completing all forms of
        authentication enumerated: iris scanning, physical security token, and passcode.
    \end{enumerate}
    \item Functional Requirement 3 (ID: AFR3) (PR1)
    \begin{enumerate}
        \item Title: Document Scanning
        \item Required For: Field Office Communication
        \item Description: The software system MUST be able to scan a variety of paper documents into a
        digital format, and only store the resulting digitized form for a maximum of one minute.
    \end{enumerate}
    \item Functional Requirement 4 (ID: AFR4) (PR1)
    \begin{enumerate}
        \item Title: Document Sending
        \item Required For: Field Office Communication
        \item Description: The software system MUST be able to send digitized documents to the desired
        field offices along a pseudo-randomized path. Document sending should be knowingly initiated and
        document receipt confirmed.
    \end{enumerate}
    \item Functional Requirement 5 (ID: AFR5) (PR2)
    \begin{enumerate}
        \item Title: Automatic Sign-Out
        \item Required For: Security
        \item Description: The software system MUST automatically sign out a field agent after a maximum of 40
        seconds of idle activity. The sign out must not be recoverable, and must only be circumvented by a full
        sign in procedure.
    \end{enumerate}
    \item Functional Requirement 6 (ID: AFR6) (PR3)
    \begin{enumerate}
        \item Title: Obfuscation
        \item Required For: Security
        \item Description: The software system’s internal workings must be obfuscated from dissection from a
        disassembler or related software programs that attempt to reverse engineer potential exploits from
        underlying machine code.
    \end{enumerate}
    \item Functional Requirement 7 (ID: AFR7) (PR2)
    \begin{enumerate}
        \item Title: Adding New Users
        \item Required For: Scalability
        \item Description: It should be a trivial process for a user with sufficient permissions to add another
        permitted user to the pool of authorized users. This new user MUST be constrained to have an authorization
        level lower than the user granting permissions.
    \end{enumerate}
\end{enumerate}

\subsubsection{User Class B - Government Officials}

\begin{enumerate}
    \item Functional Requirement 1 (ID: BFR1) (PR2)
    \begin{enumerate}
        \item Title: Server Audit
        \item Required For: FBI Accountability Mandates
        \item Description: Because security is of the utmost importance, we cannot allow API hooks within the
        application for fear that they might be misused. Therefore, the server should be able to generate a traffic
        report of the number of sensitive requests through the server since the last audit, as well as other various
        useful statistics, such as the number of times that users were automatically logged out for idling, and any
        service abnormalities that occurred.
    \end{enumerate}
    \item Functional Requirement 2 (ID: BFR2) (PR1)
    \begin{enumerate}
        \item Title: Audit Authentication
        \item Required For: FBI Accountability Mandates
        \item Description: Government officials should be subject to the same strict level of security we require for
        FBI agents to use the software. However, we do not possess retinal data for government officials. Therefore,
        government officials SHOULD be provided with a temporary physical security token on their arrival at FBI
        office premises, and they SHOULD be made to enter a secure password autogenerated at a cryptographic level
        of randomness.
    \end{enumerate}
\end{enumerate}

\subsubsection{User Class C - Foreign Operatives}

\begin{enumerate}
    \item Functional Requirement 1 (ID: CFR1) (PR1)
    \begin{enumerate}
        \item Title: Proactive monitoring
        \item Required For: Security
        \item Description: On any tampering with the physical server or upon mass document scanning from any given
        workstation, the system MUST send out an internal FBI communications message to the FBI Director, FBI IT
        manager, and the director of the specific field office being tampered with should also be notified.
    \end{enumerate}
\end{enumerate}

\subsubsection{User Class D - FBI IT Management}

\begin{enumerate}
    \item Functional Requirement 1 (ID: DFR1) (PR3)
    \begin{enumerate}
        \item Title: Documentation Portal
        \item Required For: Maintainability
        \item Description: We MUST provide an offline site for internal use by the FBI IT team, stocked with detailed
        documentation about the system from an architectural view, design view, and also construction view.
    \end{enumerate}
    \item Functional Requirement 2 (ID: DFR2) (PR2)
    \begin{enumerate}
        \item Title: Auditing System
        \item Required For: FBI Internal Self Audits
        \item Description: FBI IT team members should be able to access the same information as that provided to
        external department government auditors, and furthermore, the information should be portable from the server
        using flash drive.
    \end{enumerate}
    \item Functional Requirement 3 (ID: DFR3) (PR1)
    \begin{enumerate}
        \item Title: Internal IT Authentication
        \item Required For: FBI Internal Self Audits
        \item Description: FBI IT team members MUST provide the same level of multi-factor authentication as FBI
        agents in order to grab audit data from the server.
    \end{enumerate}
    \item Functional Requirement 4 (ID: DFR4) (PR3)
    \begin{enumerate}
        \item Title: Internal IT Audit Rates
        \item Required For: FBI Internal Self Audits
        \item Description: FBI IT team members MUST NOT be subject to any rate limiting in terms of the audits
        that they are allowed to perform. However, audits themselves should contain information about how many
        audits have been performed: an audit of audits, so to speak.
    \end{enumerate}
\end{enumerate}

\subsection{External Interface Requirements}

\subsubsection{User Interface}

The user interface should present a convenient way to scan documents after authentication, and send already
scanned documents. The sign out button should be prominently shown, and the user should be warned 10 seconds
before they are due to be signed out because of a lack of activity. The sensitive file contents being scanned
should be viewable from a separate window of the application, but should not be viewable by default from the
main entry point. The user interface should warn users if they download documents from the application to their
computer, in order to prevent leakage of sensitive information.

\subsubsection{Software Interface}

The software described herein should contain no external API hooks. It must be built from the ground up using
an FBI vetted programming language and any external library use must be on the FBI whitelist of approved
frameworks and libraries. The software interface should be modularized to allow maximum extensibility, except
in regards to authentication code, which should be left unchanged to the maximum extent in order to promote
security. On change of the software interface’s authentication methods, FBI IT management should be notified,
and the FBI director responsible for overseeing this project should also be contacted to give approval.

\subsubsection{Hardware Interface}

The hardware server must be enclosed within a secure housing inaccessible physically to personnel except for
IT specialists. It must be permanently connected to a power source that is hardened with a backup generator.

\subsubsection{Communications Interface}

The software produced must conform to the internal FBI communications specification, which will not be
enumerated here for security reasons, but can be viewed in a separate document. The communications from client
to server must be encrypted before being sent through the interface, and encryption should be AES level or higher.

\subsection{System Features}

\subsubsection{Document Transmission \& Receipt}

\begin{enumerate}
    \item Description \& Priority\\
    Priority 1: The user must be able to send scanned documents over the FBI network to other FBI offices.
    \item Stimulus \& Response Sequences
    \begin{itemize}
        \item Clicking file scan.
        \item Clicking file send after file scan complete.
        \item Having document receipt confirmed.
    \end{itemize}
\end{enumerate}

\subsubsection{Audit Generation}

\begin{enumerate}
    \item Description \& Priority\\
    Priority 2: Certain User Classes must be able to reliably obtain an audit of the system's usage and any
    abnormal behavior observed.
    \item Stimulus \& Response Sequences
    \begin{itemize}
        \item Pressing a physical button on the server.
        \item Plugging into a flash-drive port on the server.
    \end{itemize}
\end{enumerate}

\subsection{Non-Functional Requirements}

\begin{enumerate}
    \item Non-Functional Requirement 1
    \begin{enumerate}
        \item Title: Communication Scarcity
        \item Required For: Security
        \item Description: The software system MUST NOT send any extraneous information besides that
        contained within the document and that which is essential for packet routing to external FBI offices.
    \end{enumerate}
    \item Non-Functional Requirement 2
    \begin{enumerate}
        \item Title: Credentials
        \item Required For: Security
        \item Description: The software system SHOULD NOT reveal the credentials of the logged on user
        in any way to any entity other than the controlling authentication agent present in the server.
    \end{enumerate}
    \item Non-Functional Requirement 3
    \begin{enumerate}
        \item Title: Reliability
        \item Required For: Robust Software
        \item Description: The software system SHOULD remain up for at least 99.999999\% of the time.
        Failure of the central server in a given FBI field office should not affect other FBI field offices.
    \end{enumerate}
\end{enumerate}

\subsection{Design Constraints}

\begin{enumerate}
    \item Account information and case information should not be cached. Computer memory locations holding
    sensitive information should be flushed after use.
    \item Document communication must take less than one minute end-to-end, from sending to receiving at a
    field office.
    \item This software should be develop in a safe and robust programming language, Rust. Rust is also
    approved by the FBI for use in developing internal applications.
    \item Hardware should not have to be reconfigured on power failures, on maintenance downtime, and on
    office lockdown.
    \item Authentication classes should be set at the beginning of the product’s deployment, and not
    configurable short of a new software version altogether.
\end{enumerate}

\section{Change Management}

The FBI will provide vetted requirements specialists always available and internal to the agency that will
work with the contracted software company in order to ensure that the best deliverables are pursued. On any
requirements change, the signature of the FBI IT manager is required. There will be a formal meeting every
two weeks to track the progress of the software product. In order for the meeting to take place, the FBI IT
manager must be present to give comment, along with an internal auditor, the requirements specialists, and
a representative three software developers of the contracted company. After software release, the change
management procedure should go through the IT manager and the representative of the FBI Operations Department.

\section{Document Approvals}

\begin{center}
    \begin{tabular}{|c|c|c|}
        \hline
        Signatory Name & Signatory Position & Date \\ \hline
        William Fehrnstrom & IT Manager & 4/25/19 \\ \hline
        Elizabeth Han & Software Developer & 4/25/19 \\ \hline
        Michael Wang & Software Developer & 4/25/19 \\ \hline
        Michael Wu & Software Developer & 4/25/19 \\ \hline
    \end{tabular}
\end{center}

\pagebreak
\part{Glucose Monitor Wireless Data Transfer}

\section{Introduction}

\subsection{Purpose}

The purpose of this software is the following: transfer glucose monitoring data to a remote database. Diabetic
patients are normally given monitors to track their blood sugar level. However, all the collected data is stored
locally, which makes it difficult for medical personnel to access when they need in order for an accurate diagnosis.
The files would have to be manually sent over by the client, which could induce a time delay in the analysis the
doctor would produce after looking over the data. A tightly controlled server will need to be introduced to
automatically send the data in real time.

\subsection{Scope}

This product will focus on the sharing of locally stored data on continuous glucose monitoring systems, as well
as the authentication to make sure that patients have control over who is accessing their data. This product will not
focus on analysis of the actual glucose data beyond the very basic, i.e. glucose levels being dangerously high.

\subsection{Intended Audience and Use}

The diabetic patients and the physicians/care providers. Patients should be able to view their real-time and stored data
-- physicians and caregivers may view this data after patients give permissions.

\subsection{Acronyms and Abbreviations}

\begin{enumerate}
    \item Patient: a diabetic using glucose monitoring software in order to maintain their health
    \item Physicians/Care providers: the medical personnel in charge of monitoring the patient's health and providing
    their professional opinion and diagnoses
    \item Development Team: The contracted development team working on this product
    \item HTTP: Hypertext Transfer Protocol.
    \item TCP: Transmission Control Protocol.
    \item POST Request: A request made to a server to update its data.
    \item GET Request: A request made to a server to retrieve information from it.
\end{enumerate}

\section{Overall Description}

\subsection{Product Background and Perspectives}

Continuous glucose monitoring systems already provide a way for patients to see their glucose levels in real time, as
well as view trends. However, they do not possess the capability to let physicians view this data.  Patient privacy
is of great importance: only medical personnel assigned to the specific patient should be able to view this data,
and only if the patient gives continuous permission for them to see it.  The one exception is if the patient exceeds
acceptable glucose levels; the aforementioned medical personnel will NEED to see this information to ensure the patient's health.

\subsection{User Needs}

\begin{center}
    \begin{tabularx}{\linewidth}{|c|X|}
        \hline
        Patient & This user class comprises all of the FBI agents that will be using our product in the field to help
        distribute case information between offices.\\
        \hline
        Physician & This user class consists of all the doctors who will be viewing glucose data and receiving it on a
        consistent basis. \\
        \hline
    \end{tabularx}
\end{center}

\subsection{Assumptions and Dependencies}

We assume that the patient is in possession of a glucose monitoring device.  We assume that any user that logs in is,
in fact, that user: it is patient and physician responsibility to keep their usernames/passwords private. We also assume
that the physician always has internet connection, courtesy of their place of employment.  We assume that the client will
have internet connection at least once every 24 hours.

\subsection{Constraints}

\begin{enumerate}
    \item The patient must give their consent to allow the physician to access their personal glucose data, otherwise
    the patient's glucose data must not be revealed to any third party without their knowledge.
    \item The system must take some action given a patient's high glucose level.
    \item The glucose monitoring device has only 10 Megabytes of storage.
    \item The patient may sometimes enter areas without a network connection.
\end{enumerate}

\section{System Features \& Requirements}

\subsection{Functional Requirements}

\subsubsection{User Class A - General}

\begin{enumerate}
    \item Functional Requirement 1 (ID: AFR1) (PR1)
    \begin{enumerate}
        \item Title: Transfer
        \item Required For: Data Transfer to Physician
        \item Description: If an internet connection is available, all measurements taken by the glucose monitor
        shall be automatically transferred wirelessly to a centralized database. This information is linked with
        the patient’s account and is usually only viewable by the patient unless the patient explicitly shares
        this information or under extreme circumstances.
    \end{enumerate}
    \item Functional Requirement 2 (ID: AFR2) (PR1)
    \begin{enumerate}
        \item Title: Storage
        \item Required For: Completeness of Data
        \item Description: If an internet connection is unavailable, the glucose monitor shall store the most recent
        readings locally, discarding the oldest readings if the memory limit is exceeded. Upon reconnecting to the
        internet, the glucose monitor shall prioritize sending the oldest readings on record.
    \end{enumerate}
    \item Function Requirement 3 (ID: AFR3) (PR1)
    \begin{enumerate}
        \item Title: Analysis
        \item Required For: Preliminary Warning
        \item Description: The database shall be capable of performing a preliminary analysis of the readings such
        as detecting life threatening readings or trends, but this preliminary analysis should not go beyond basic
        detection and notification.
    \end{enumerate}
\end{enumerate}

\subsubsection{User Class B - Patient}

\begin{enumerate}
    \item Functional Requirement 1 (ID: BFR1) (PR1)
    \begin{enumerate}
        \item Title: Permissions
        \item Required for: Legal Concerns
        \item Description: From their accounts, patients shall be able to give access to their readings based on
        date of measurement to their care provider and physician of choice. Patients can choose additional physicians
        and care providers at any time.
    \end{enumerate}
    \item Functional Requirement 2 (ID: BFR2) (PR1)
    \begin{enumerate}
        \item Title: Personal Account
        \item Required for: Ease of Use
        \item Description: Patients shall be able to log on to to a personal account to view all past readings.
        UI elements such as graphs shall be provided to aid in the visualization of the data.
    \end{enumerate}
    \item Functional Requirement 3 (ID: BFR3) (PR2)
    \begin{enumerate}
        \item Title: Inoperability Notice
        \item Required for: General use
        \item Description: Patients shall know when their devices are malfunctioning via device notifications.
        If malfunctions inhibit the device notifications, the device should perform a soft shut down.
    \end{enumerate}
\end{enumerate}

\subsubsection{User Class C - Physicians/Care providers}

\begin{enumerate}
    \item Functional Requirement 1 (ID: CFR1) (PR1)
    \begin{enumerate}
        \item Title: Permissions
        \item Required for: Legal Concerns
        \item Description: Physicians and care providers shall not have access to patient measurements unless given
        explicit permission by the patient or unless the readings meet certain criteria (defined by the monitor
        manufacturer) indicating the patient’s health to be in danger. Readings that meet these criteria shall be
        automatically shared with the patient’s physician and care provider even without explicit permission.
    \end{enumerate}
    \item Functional Requirement 2 (ID: CFR2) (PR1)
    \begin{enumerate}
        \item Title: Patient Alert
        \item Required for: Real-Time Wellness
        \item Description: Physicians and care providers should be notified when a patient's glucose level spikes
        to dangerous levels. They should be able to be notified through the following communication channels: email and text.
   \end{enumerate}
   \item Functional Requirement 3 (ID: CFR3) (PR3)
    \begin{enumerate}
        \item Title: Patient History
        \item Required for: Data Review
        \item Description: Physicians and care providers should be able to view a graph of patient glucose levels over
        time in real-time.
   \end{enumerate}
\end{enumerate}

\subsection{External Interface Requirements}

\subsubsection{User Interface}

A UI should be present on the display of the patient's glucose monitor and should inform the user of high glucose levels,
as well indicate when the system has become inoperable in an alarming font and icon. There should be a UI for viewing
glucose data on a web portal. On the Web Portal, patients will be given a prompt to sign in to be able to use the
system. Once they have signed in, they should immediately see any alarm notifications as cards underneath a graph
displaying their glucose levels. The physician should have access to the same UI, once granted permission by the patient.

\subsubsection{Software Interface}

The software interface provided by the glucose monitoring device should contain operations for retrieving data in a given
span of time, and retrieving data points above a certain threshold. The web portal should operate on an analysis API run
from a remote server that processes user glucose levels and performs common analytics on them.

\subsubsection{Hardware Interface}

The hardware interface given by the glucose monitoring device should allow for networked communication, temporary
storage of glucose data, and should be operable under operating conditions from \(-10\degree\)F to \(130\degree\)F.

\subsubsection{Communications Interface}

The Communication Interface for the glucose monitor shall be TCP. HTTP POST/GET requests will be made to the analytics
server from client computers (physicians and patients)

\subsection{System Features}

\subsubsection{Data Transmission}

\begin{enumerate}
    \item Description \& Priority\\
    Priority 1: Users must be able to access real-time records of the glucose levels of the patient
    \item Stimulus \& Response Sequences
    \begin{itemize}
        \item Log into account
        \item (if physician) Select patient name
        \item Click view records
        \item Have records displayed, possibly in graphical format
    \end{itemize}
\end{enumerate}

\subsubsection{Privacy}

\begin{enumerate}
    \item Description \& Priority\\
    Priority 2: Patients must be able to reliably control who is able to access their data
    \item Stimulus \& Response Sequences
    \begin{itemize}
        \item Automatic notification after a certain period of time, asking patient to give permission for
        physician to view data
        \item Click 'Yes' or 'No'
        \item Have permissions confirmed or denied
    \end{itemize}
\end{enumerate}

\subsection{Non-Functional Requirements}

\begin{enumerate}
    \item The service must have medical rated reliability: uptime must be more than 99.9999999\%
    \item The service must have reasonable response time: under normal operating conditions, it should take a maximum
    of a minute from a new glucose data point to server processing. Under dangerous conditions, where a user is in a
    life threatening situation, system total response time must be under 15 seconds.
    \item The service must be secure: multi-factor authentication should be used. At the very least, strong
    password protection must be mandated for every user.
\end{enumerate}

\section{Change Management}

At the beginning of the development process, focus groups of physicians and glucose sensitive patients should be
assembled, and asked to participate in a semi-regular product feedback loop during the development process. These
product feedback meetings should take place every month at the end of an agile sprint. Developers should program in pairs.

\section{Document Approvals}

\begin{center}
    \begin{tabular}{|c|c|c|}
        \hline
        Signatory Name & Signatory Position & Date \\ \hline
        Elizabeth Han & Medicorp Director & 4/25/19 \\ \hline
        Will Fehrnstrom & Software Developer & 4/25/19 \\ \hline
        Michael Wang & Software Developer & 4/25/19 \\ \hline
        Michael Wu & Software Developer & 4/25/19 \\ \hline
    \end{tabular}
\end{center}

\pagebreak
\part{Windows Upgrade Application for Bank}

\section{Introduction}

\subsection{Purpose}

The client is a bank that has spent money developing software that works on Windows 7. Our client
wishes to create an application that helps upgrade their code to be compatible with Windows 10.
This should be done safely, easily, and automatically so that the client has minimal hassle with the upgrade
process.

\subsection{Scope}

Our application will be an IDE that performs automatic code conversion, formatting, and and refactoring. It will also
generate diffs in order to help the bank know what has been changed. We will add tools
to refactor various parts of the code such as changing variable names and function names.

\subsection{Intended Audience and Use}

The two main types of people who will be dealing with our software will be the bank's developers and the bank's testers.
The developers want to have an easy to use editor that changes code quickly and error free, while the testers want
an easy way to see what has been changed and why. The developers will be able to edit in a graphical environment
that includes a ``convert'' function which automatically detects and fixes any code sections that are incompatible with Windows 10.
The testers will be able to look at logs generated by the ``convert'' functionality that indicates what was changed.

\subsection{Acronyms and Abbreviations}

\begin{enumerate}
    \item IDE: Integrated Development Environment
    \item UI: User Interface
    \item Diff: A log that shows the difference between a file before and after changes
    \item Client: The bank
\end{enumerate}

\section{Overall Description}

\subsection{Product Background and Perspectives}

Our application makes it easy to convert code that works on Windows 7 to Windows 10. It automatically detects any outdated
code constructs and replaces them with workarounds that are compatible with Windows 10. This enables developers to quickly
get their applications working on the new platform with minimal hassle. The code conversion process should only take a
single click for the entire project. Developing this system will save our clients the cost of having humans
manually converting every file in a large code base.

\subsection{User Needs}

\begin{center}
    \begin{tabularx}{\linewidth}{|c|X|}
        \hline
        Developers & This user class comprises all of the bank's employees that will be using our product to convert their code to
        use Windows 10 compatible code.\\
        \hline
        Testers & This user class consists of the bank's employees that will be looking at the logs that our product generates and verifying
        that the code is compatible with Windows 10.\\
        \hline
    \end{tabularx}
\end{center}

\subsection{Assumptions and Dependencies}

An assumption that we make is that the code is easily readable by a machine and able to be converted to Windows 10. Sometimes
code upgrades require sophisticated workarounds that only a human can perform. In this case our product will simply show
an error message that a conversion could not be performed. Then a human can fix the problem. Additionally we assume that
the main issues with the upgrade are code incompatibilities due to different functions, naming conventions, and other constructs
that can be parsed. Then our IDE can use algorithms similar to those used in compilers to figure out what code constructs
are being used. We assume that our client's application uses standard object oriented coding languages such as those related to
Java and C++, so we do not need to develop a converter for every possible language. We can expand the covered languages at a later date.

\subsection{Constraints}

\begin{enumerate}
    \item We are given a set of source files and must produce an updated set of source files that is compatible with Windows 10.
    \item The input source files are compatible with Windows 7.
    \item Sometimes code changes cannot be converted to be compatible and our system must detect this.
    \item The system must work on the languages that are used in our client's applications.
\end{enumerate}

\section{System Features \& Requirements}

\subsection{Functional Requirements}

\textbf{Priorities are ranked from level 1 (PR1) (imperative) to level 3 (PR3) (Add-On Feature)}

\subsubsection{User Class A: Developers}

\begin{enumerate}
    \item Functional Requirement 1 (ID: AFR1) (PR1)
    \begin{enumerate}
        \item Title: Convert
        \item Required For: Automated Changes
        \item Description: The IDE should look through the source code and swap out any incompatible code
        with new code that works with Windows 10.
    \end{enumerate}
    \item Functional Requirement 2 (ID: AFR2) (PR2)
    \begin{enumerate}
        \item Title: Refactor
        \item Required For: Manual Changes
        \item Description: The IDE should be able to perform actions such as renaming variables, changing function names, modifying
        function definitions, formatting code, and rearranging functions so that developers can easily make manual code changes.
    \end{enumerate}
    \item Functional Requirement 3 (ID: AFR3) (PR1)
        \begin{enumerate}
        \item Title: Error Detection
        \item Required For: Manual Changes
        \item Description: The IDE should be able to highlight any incompatible code sections that cannot be converted so that
        developers can see where they need to make manual changes.
    \end{enumerate}
    \item Functional Requirement 4 (ID: AFR4) (PR3)
        \begin{enumerate}
        \item Title: Source Explorer
        \item Required For: Ease of Use
        \item Description: The IDE should be able to show a tree that contains the source files in the project. This allows developers to
        easily navigate through the code base.
    \end{enumerate}
\end{enumerate}

\subsubsection{User Class B: Testers}

\begin{enumerate}
    \item Functional Requirement 1 (ID: BFR1) (PR2)
    \begin{enumerate}
        \item Title: Diffs
        \item Required For: Logging
        \item Description: The IDE should generate a diff that shows changes made after converting code. This way testers can verify that
        the code works and add comments to a change log.
    \end{enumerate}
    \item Functional Requirement 2 (ID: BFR2) (PR2)
    \begin{enumerate}
        \item Title: Help Messages
        \item Required For: Logging
        \item Description: The IDE should be able to explain any changes with a help message.
    \end{enumerate}
\end{enumerate}

\subsection{External Interface Requirements}

\subsubsection{User Interface}

The UI will be a graphical interface that displays the project structure of the code that a developer is working on. There should
also be an editor window that lets the developer make changes to the source code. There will be a menu option to convert the code
to be compatible with Windows 10.

\subsubsection{Software Interface}

Our software will be a standalone application that runs locally on the client's machines. Because of this, there will be no API calls
to any external sources or any other software components apart from the IDE executable.

\subsubsection{Hardware Interface}

Our software will be compiled to run of the machines that our client uses for development. As our application only runs on a single
machine at a time, no other hardware support is needed. We may need two versions of our IDE, but only if our client specifies that
they develop on two incompatible platforms such as Linux and Windows.

\subsubsection{Communications Interface}

Our IDE will initially not perform any communication, though we may later add in a feature to perform updates over the network.

\subsection{System Features}

\subsubsection{Code Conversion}

\begin{enumerate}
    \item Description \& Priority\\
    Priority 1: The user must be able to automatically convert code to be compatible with Windows 10.
    \item Stimulus \& Response Sequences
    \begin{itemize}
        \item Load project into IDE.
        \item Click to convert code to be compatible with Windows 10.
        \item Show a results screen with any possible errors or success status.
    \end{itemize}
\end{enumerate}

\subsubsection{Log Generation}

\begin{enumerate}
    \item Description \& Priority\\
    Priority 2: After the conversion some sort of log should be generated.
    \item Stimulus \& Response Sequences
    \begin{itemize}
        \item The developer clicks to convert code.
        \item A log file is saved in the file system with a diff and descriptions of changes.
    \end{itemize}
\end{enumerate}

\subsection{Non-Functional Requirements}

\begin{enumerate}
    \item Non-Functional Requirement 1
    \begin{enumerate}
        \item Title: Execution Time
        \item Required For: Speed
        \item Description: The IDE should be able to update the code in \(O(n^2)\) time or better in order to ensure that the code updates
        take a reasonable amount of time.
    \end{enumerate}
    \item Non-Functional Requirement 2
    \begin{enumerate}
        \item Title: Multiple Language Support
        \item Required For: Flexibility
        \item Description: The IDE should be able to convert code in multiple languages so that all the code in the client's applications
        can be converted.
    \end{enumerate}
    \item Non-Functional Requirement 3
    \begin{enumerate}
        \item Title: Simple Interface
        \item Required For: Ease of Use
        \item Description: The IDE should be easy to navigate and learn so that developers can quickly begin to use it.
    \end{enumerate}
\end{enumerate}

\subsection{Design Constraints}

\begin{enumerate}
    \item We will develop the application as a lightweight standalone application. This way it can be quickly distributed.
    \item The application should focus on the minimum functionality in order to get the conversion process started quickly.
    \item We will design tokenizers and parsers for each language and create an abstract syntax tree in order to process the code.
    \item We will research the code differences in Windows 10 and Windows 7 in order to make the appropriate transformations on the code.
    These changes will be hard coded into our application.
\end{enumerate}

\section{Change Management}

Changes will go through the client first. For each change we will have a meeting and
draft a change proposal in order to document the reasons why we are making a change. During
our change meetings we will also specify the additional costs of making a change.

\section{Document Approvals}

\begin{center}
    \begin{tabular}{|c|c|c|}
        \hline
        Signatory Name & Signatory Position & Date \\ \hline
        Michael Wu & Client & 4/25/19 \\ \hline
        Elizabeth Han & Software Developer & 4/25/19 \\ \hline
        Michael Wang & Software Developer & 4/25/19 \\ \hline
        William Fehrnstrom & Software Developer & 4/25/19 \\ \hline
    \end{tabular}
\end{center}

\end{document}